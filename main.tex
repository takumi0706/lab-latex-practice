\documentclass{classes/report}
\usepackage{tocloft} % 目次関連のパッケージ
\usepackage{fancyhdr}
\usepackage{enumitem}
\usepackage{chngcntr} % カウンターの依存関係を変更するパッケージ

% 表のキャプション名を「表」に変更
\renewcommand{\tablename}{表}
% 表番号から章番号を省いて連番にする
\counterwithout{table}{chapter}

% 脚注の番号を非表示にする設定
\makeatletter
\renewcommand\@makefntext[1]{\noindent\makebox[0pt][r]{\@thefnmark\hspace{.2em}}\@setfontsize\footnotesize\@ixpt{11}#1}
\renewcommand\@makefnmark{}
\makeatother

% @を含むコマンドを修正するためのパッチ
\makeatletter
\def\@chapapp{第}  % 明示的に\@chapappを定義する
\makeatother

\begin{document}

\pagenumbering{roman}
\subfile{sections/title}
\newpage

\renewcommand{\contentsname}{\Large 目次} % 目次のタイトルを小さく

\setlength{\cftbeforetoctitleskip}{0pt} % 目次タイトル前の空白をなくす
\setlength{\cftaftertoctitleskip}{0pt} % 目次タイトル後の空白を調整

\renewcommand{\cftsubsecpagefont}{\normalfont} % 目次のサブセクションのページ番号を通常フォントに
\renewcommand{\cftsubsubsecpagefont}{\normalfont} % 目次のサブサブセクションのページ番号を通常フォントに

\tableofcontents
\clearpage

\pagenumbering{arabic}

\subfile{sections/first}

\subfile{sections/attention}

\subfile{sections/account_password}

\subfile{sections/application}

\subfile{sections/commandLine}

\subfile{sections/unix}

\subfile{sections/textEditer}

\subfile{sections/createDocment}

\subfile{sections/createFigure}

\subfile{sections/useOutStorage}

\subfile{sections/useWirelessNetwork}

\end{document}