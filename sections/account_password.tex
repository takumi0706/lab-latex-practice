%! TEX root = ../main.tex
\documentclass[main]{subfiles}

\begin{document}
\vspace{30pt}
\chapter{アカウントとパスワード}

\fontsize{10}{15} \selectfont
Unix計算機を使用するには,その権利である\textbf{アカウント}と認証用の\textbf{パスワード}が必要です.\\
\indent
パスワードは,英数字と記号(!,#,$など)からなる文字列であり,多い分には何文字あっても構いませんが,認証に用いられるのは最初の8文字です.\\
\indent
パスワードの目的は,認証されてはならない人物によるシステムへのアクセスを防止することですから,相当の努力をもってしても予測し難いものを選ばなければなりません.\textbf{クラッカー}にパスワードが破\\

\fontsize{10.2}{15} \selectfont
\noindent
られると,破られたユーザの被害だけでは済まず,それが糸口となって他のユーザやシステム全体にまで被害が及びます.したがって,ユーザは\textbf{自分のためというよりむしろ他のユーザのために}責任をもって自分のアカウントを管理しなければなりません.また,次のようなパスワードを破るのは容易であると言われています\textsuperscript{2}.\\

\begin{itemize}
    \setlength{\itemsep}{0.7zw}         % 各項目の間隔
    \item 自分の名前,アカウント名,誕生日,車のナンバー,電話番号,学籍番号,住所,所属などの個人情報.またはこれらの一部.個人情報は自分以外のものでも避けるべきである.
    \item 辞書に載っている単語,辞書に載っていなくても誰でも知っている単語(略語,スラング,何かのスローガンなど),有名人,スポーツのチーム名,バンド名,映画,テレビ,小説,マンガ等の作品名や登場人物.
    \item 同じ文字の繰り返し,数字だけからなるもの.
    \item キーボードの隣になったキーを押したもの.
    \item 以上挙げたものの反転文字列.
\end{itemize}

パスワードをメモに残すのもよくないことです.そのメモが新たな\textbf{セキュリティーホール}になるからです.\\

\footnote{\textsuperscript{2} 実際,当研究室の環境では,あまりに単純なパスワードは拒否されます.パスワードは十分な複雑さを持たせるようにしてください.}


\end{document}