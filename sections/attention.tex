%! TEX root = ../main.tex
\documentclass[main]{subfiles}
\usepackage{enumitem}

\begin{document}
\chapter{注意事項}

\fontsize{10}{10} \selectfont
\noindent
\textbf{電源について} 帰宅時にディスプレイの電源は落としてください.ただしUnixは24時間稼働が基本ですの\\
\hspace{2em}で,計算機本体の電源は落とさないでください.\\

\noindent
\textbf{計算機本体にはやさしく} ハードディスクの寿命を知事めることのないよう,本体に衝撃を与えることは避\\
\hspace{2em}けてください.\\

\noindent
\textbf{ディスプレイはきれいに}  ディスプレイに直接指で触れることは避けてください.指の脂が付着して画面\\
\hspace{2em}が見えにくくなってしまうからです.誤って触ってしまった時は自分で拭き取ってください.他の人\\
\hspace{2em}に画面の内容を示す時はマウスポインタを使ってください.\\

\noindent
\textbf{キーボードに食べ物を与えない} 計算機の周囲で飲み食いするのは構いませんが,その際はキーボードに与\\
\hspace{2em}えてしまうことのないように細心の注意を払ってください.特にカップラーメン,コーヒーお茶の\\
\hspace{2em}類を与えると一発であの世行きとなってしまいます.\\

\noindent
\textbf{プリンタ用紙について} 原則として,プリンタ用紙は以下のように使い分けてください.
\begin{itemize}
    \renewcommand{\labelitemi}{}
    \setlength{\itemsep}{0.7zw}         % 各項目の間隔
    \item \textbf{未使用再生紙(白色度80%)} 卒論・修論最終版.印刷前に教員まで取りに来てください.
    \item \textbf{未使用再生紙(白色度70%)} 通常印刷.
    \item \textbf{片面使用済み} 計算用紙など.プリンター内部の劣化を早める恐れがあるので印刷用には使用しないでください.
    \item 未使用再生紙がなくなったら教員まで取りに来てください.
\end{itemize}

\noindent
\textbf{トラブルに遭遇したら} 何かのトラブルに遭遇したら,そのままの状態にして速やかに管理者に報告し\\
\hspace{2em}てください.ただしプリンタの印刷が止まらなくなった場合は紙の節約のため,\textbf{直ちに給紙カー}\\
\hspace{2em}\textbf{トリッジを引き抜く}こと.
\end{document}