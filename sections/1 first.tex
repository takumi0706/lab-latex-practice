%! TEX root = ../main.tex
\documentclass[main]{subfiles}

\begin{document}

\chapter{はじめに}

この文書では,当研究室の計算機を利用する方法・注意等を簡単にまとめています.表\ref{tab:computers}は現在の当研究室の計算機の一覧です.\\
\indent
LinuxはUnixではありませんが,本文書が扱う範囲で両者の区別が必要になることはないので,以後Unix系OSの総称としてUnixという言葉を使用します.\\
\indent
原則として,キーストロークはEmacs風に表記します.例えば,\texttt{RET}は\roundedbox{ \texttt{Enter} },\texttt{S-a}は\roundedbox{ \texttt{Shift} }を押しながら\roundedbox{\texttt{a}},\texttt{C-SPC}は\roundedbox{ \texttt{Ctrl} }を押しながら\roundedbox{ \texttt{Space} },
\texttt{M-x}は左\roundedbox{\texttt{Alt}}を押しながら\footnote{\texttt{M-x}で\roundedbox{\texttt{x}}とともに押すキーはMetaキーと呼ばれ,AT互換機では通常左\roundedbox{\texttt{Alt}}がMetaキー割り当てられています.}\roundedbox{\texttt{x}}を押すことをそれぞれ意味します.\\

\begin{table}[p]
    \caption{当研究室の学生用計算機一覧(2024年4月12日現在)}
    \label{tab:computers}
    \begin{center}
    \begin{tabular}{|l|l@{\hspace{0.5em}}l@{\hspace{1em}}l|l|}
    \hline
    メーカ型名 & \multicolumn{3}{l|}{スペック} & ホスト名 \\
    \hline
    Dell Vostro 460 
     & CPU:    & Intel Core i7-2600 3.40GHz (4 cores) & & marlin \\
     & メモリ: & 8GB & & \\
     & HDD:    & 500GB & & \\
     & OS:     & Debian GNU/Linux 10.9 & & \\
     &         & Windows 7 Professional & & \\
    \hline
    Dell Precision T5500 
     & CPU:    & Intel Xeon E5620 2.40GHz (4 cores) & & shelly \\
     & メモリ: & 8GB & & \\
     & HDD:    & 500GB×2 & & \\
     & OS:     & Debian GNU/Linux 10.9 & & \\
     &         & Windows 7 Professional & & \\
    \hline
    Dell PowerEdge T110 
     & CPU:    & Intel Xeon X3450 2.66GHz (8 cores) & & genie \\
     & メモリ: & 8GB & & \\
     & HDD:    & 1TB×2 (RAID 1) & & \\
     & OS:     & Debian GNU/Linux 10.9 & & \\
    \hline
    Dell Precision T5810 
     & CPU:    & Xeon E5-1680 3.40GHz (6 cores) & & saleen \\
     & メモリ: & 64GB & & \\
     & HDD:    & 2TB×2 & & \\
     & OS:     & Debian GNU/Linux 10.9 & & \\
    \hline
    Dell Precision T3600 
     & CPU:    & Xeon E5-1660 3.30GHz (6 cores) & & diablo \\
     & メモリ: & 64GB & & \\
     & HDD:    & 2TB×2 & & \\
     & OS:     & Debian GNU/Linux 10.9 & & \\
     &         & CentOS & & \\
    \hline
    Dell Dimension 4100 
     & CPU:    & PentiumIII 1GHz & & carrera \\
     & メモリ: & 256MB & & \\
     & HDD:    & 250GB & & \\
     & OS:     & Debian GNU/Linux 10.9 & & \\
    \hline
    \end{tabular}
    \end{center}
\end{table}

\end{document}