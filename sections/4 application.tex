%! TEX root = ../main.tex
\documentclass[main]{subfiles}

\begin{document}
\chapter{アプリケーション}

基本的に,画面左上の「アクティビティ」→「アプリケーション」メニューから選んで実行します.以下で,使用するアプリケーションについていくつか注意する点を挙げます.

\begin{description}
    \item[デスクトップ環境] GNOME と呼ばれるデスクトップ環境を採用しています.

    \item[日本語エンコーディング] 日本語の文字エンコーディングにはいくつか種類があります.当研究室ではUTF-8をデフォルトに設定しています.

    \begin{description}

        \vspace{0.7em}

        \item[\textrm{EUC}] 以前Unix で広く採用されていたコード.

        \item[\textrm{JIS(ISO-2022-JP)}] E-mailで採用されているコード

        \item[\textrm{SJIS}] Microsoft, Apple等商用PCでよく採用されているコード

        \item[\textrm{UTF-8}] 現在の世界標準である Unicode と呼ばれるコードの一部.MS-Windows の日本語ファイル名等にも使われている.
        
    \end{description}
    
    \item[端末エミュレータ] コマンドライン(古語)を使用するためのアプリケーション.普通は GNOME 端末(gnome-terminal)を使ってください.JIS と UTF-8(または EUC 他;超漢字等非対応)に対応しています.

    \item[テキストエディタ] 少なくとも UTF-8 を扱えるものを使ってください.

    \begin{description}

        \vspace{0.7em}

        \item[\textrm{Emacs(+yatex)}] 上級者向けだが,C や \LaTeX のソースの入力サポートが高機能.7面で解説.
    
    \end{description}

    \item[\textrm{Web} ブラウザ]  

    
    \begin{description}

        \vspace{0.7em}

        \item[\textrm{Firefox}] Mozilla 製ブラウザ.
    
    \end{description}

    \item[メーラー]  

    \begin{description}

        \vspace{0.7em}

        \item[\textrm{Thunderbird}] 当研究室ではこれを推奨.初めて使う時は次のように設定してください:

        \begin{enumerate}[itemsep=0.8em]
            
            \item Thunderbirdを起動して現れるポップアップで「メールアカウントを設定する」をクリック
            
            \item 以下を設定して「続ける」をクリック

                \begin{itemize}
                    
                    \item あなたのお名前: (例) Akira Matsubayashi

                    \item メールアドレス: (例) mbayashi@genie.ec.t.kanazawa-u.ac.jp

                    \item パスワード: (ログインパスワード)
                
                \end{itemize}

            \item 以下が検出されるのを確認して「完了」

                \begin{itemize}
                    
                    \item 受信サーバ: POP3, genie.ec.t.kanazawa-u.ac.jp, STARTTLS

                    \item 送信サーバ: SMTP, genie.ec.t.kanazawa-u.ac.jp, 接続の保護なし

                    \item ユーザ名: (例) mbayashi
               
                \end{itemize}

            \item 「送信サーバ設定...接続時暗号化されません」の警告文で「接続するときの危険性を理解しました」をチェックして「完了」

            \item 「セキュリティ例外の追加」ポップアップで「セキュリティ例外を承認」

        \end{enumerate}

        \item [\textrm{Emacs+Mew}] メール本文の作成・編集に Emacs を使いたい人向け.1 通あたり1ファイルに分かれるので,Unix コマンドの機能をフルに使って検索・整理・分類ができるという利点がある.

    \end{description}

    \item[\textrm{PDF} ビューア] PDFファイルのビューアです.  

    \begin{description}
        
        \item[\textrm{Adobe Reader (acroreader)}] Adobe 社の純正ビューア.動作が重いが表示品質は良い.印刷にはこちらを使う.

        \item[\textrm{evince}] 表示品質は良くないが若干動作が軽い.ちょっと見る時には便利.PostScript ファイルを見ることもできます.
    
    \end{description}

    \item[数学関連]  

    \begin{description}

        \item[\textrm{gnuplot}] 関数およびデータの2/3次元プロッタ.

        \item[\textrm{maxima}] Mathematica 風の数式処理パッケージ.
        
    \end{description}

    \item[ドローツール] 図を作成するツールです.EPS を作成できるものを使ってください.

    \begin{description}
       
        \item[\textrm{Inkscape}] SVG 編集ソフト.基本的にはこちらを推奨.

        \item[\textrm{LibreOffice Draw (libreoffice)}] もう一つの選択肢.
   
    \end{description}

    \item[フォトレタッチツール] gimp があります.
    
    \item[オフィスアプリケーション] LibreOffice シリーズがあります.

\end{description}
\end{document}