%! TEX root = ../main.tex
\documentclass[main]{subfiles}
\renewcommand{\thefootnote}{} % 脚注の番号を非表示に変更

\begin{document}

\chapter{はじめに}

\fontsize{10.2}{15} \selectfont
この文書では,当研究室の計算機を利用する方法・注意等を簡単にまとめています.表1は現在の当研究室の計算機の一覧です.\\
 LinuxはUnixではありませんが,本文書が扱う範囲で両者の区別が必要になることはないので,以後Unix系OSの総称としてUnixという言葉を使用します.\\
 原則として,キーストロークはEmacs風に表記します.例えば,RETは\roundedbox{Enter},S-aは\roundedbox{Shift}を押しながら\roundedbox{a},C-SPCは\roundedbox{Ctrl}を押しながら\roundedbox{Space},M-xは左\roundedbox{Alt}を押しながら\roundedbox{x}を押すことをそれぞれ意味します.\\
\footnote{\textsuperscript{1}M-xで\roundedbox{x}とともに押すキーはMetaキーと呼ばれ,AT互換機では通常左\roundedbox{Alt}がMetaキー割り当てられています.}

\begin{table}[p]
    \caption{当研究室の学生用計算機一覧(2024年4月12日現在)}
    \begin{center}
    \begin{tabular}{|l|l|l|}
    \hline
    メーカ型名 & スペック & ホスト名 \\
    \hline
    Dell Vostro 460 & CPU: Intel Core i7-2600 3.40GHz (4 cores) & marlin \\
     & メモリ: 8GB & \\
     & HDD: 500GB & \\
     & OS: Debian GNU/Linux 10.9 & \\
     &     Windows 7 Professional & \\
    \hline
    Dell Precision T5500 & CPU: Intel Xeon E5620 2.40GHz (4 cores) & shelly \\
     & メモリ: 8GB & \\
     & HDD: 500GB×2 & \\
     & OS: Debian GNU/Linux 10.9 & \\
     &     Windows 7 Professional & \\
    \hline
    Dell PowerEdge T110 & CPU: Intel Xeon X3450 2.66GHz (8 cores) & genie \\
     & メモリ: 8GB & \\
     & HDD: 1TB×2 (RAID 1) & \\
     & OS: Debian GNU/Linux 10.9 & \\
    \hline
    Dell Precision T5810 & CPU: Xeon E5-1680 3.40GHz (6 cores) & saleen \\
     & メモリ: 64GB & \\
     & HDD: 2TB×2 & \\
     & OS: Debian GNU/Linux 10.9 & \\
    \hline
    Dell Precision T3600 & CPU: Xeon E5-1660 3.30GHz (6 cores) & diablo \\
     & メモリ: 64GB & \\
     & HDD: 2TB ×2 & \\
     & OS: Debian GNU/Linux 10.9 & \\
     &     CentOS & \\
    \hline
    Dell Dimension 4100 & CPU: PentiumIII 1GHz & carrera \\
     & メモリ: 256MB & \\
     & HDD: 250GB & \\
     & OS: Debian GNU/Linux 10.9 & \\
    \hline
    \end{tabular}
    \end{center}
\end{table}

\end{document}